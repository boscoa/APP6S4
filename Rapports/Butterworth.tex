\documentclass{article}
\usepackage[utf8]{inputenc}
\usepackage{amsmath}
\usepackage{graphicx}
\usepackage{geometry}
\usepackage[english, french]{babel}
\graphicspath{{images/} 
\geometry{legalpaper, lmargin=0.7in, bmargin=1in}}
\selectlanguage{french}

\setlength\parindent{0pt}% globally suppress indentation

\begin{document}
%%%%%%%%%%%%%%
%page  titre en caractères plus large
%%%%%%%%%%%%%%
\begin{titlepage}   
	\large{
		\begin{center}
			UNIVERSITÉ DE SHERBROOKE\\Faculté de génie\\
			Département de génie électrique et génie informatique\\
			\vspace{3cm}
			{\LARGE\textbf{Traitement numérique des signaux III}}\\
			\vspace{2cm}
			\LARGE{Rapport APP6}\\
			\vspace{2cm}
			Présenté à\\l'équipe professorale de la session S4\\
			\vspace{2cm}
			Produit par\\Éric Beaudoin - beae2211\\Axel Bosco - bosa2002\\Jacob Fontaine - fonj1903\\Philippe Spino - spip2401\\
			\vspace{1cm}
			\vfill{26 juillet 2017 - Sherbrooke}
		\end{center}
	}
\end{titlepage}

\newpage
\section{Design d'un des filtres Butterworth}
Pour la conception du filtre Butterworth à partir d'un filtre passe-bas, on doit identifier l'ordre de celui-ci, dans notre cas, d'ordre 1. On choisit donc la représentation générale de la fonction de transfert d'un filtre passe-bas d'ordre 1, soit:
\begin{equation}
H(s)=\frac{1}{s+1}
\end{equation}
En utilisant la valeur de s lorsqu'on passe d'un filtre passe-bas à un filtre passe-bande, soit:
\begin{equation}
s = \frac{s^2+\omega_a\omega_b}{(\omega_b-\omega_a)s}
\end{equation}
On obtient:
\begin{equation}
H(s)=\frac{1}{\frac{s^2+\omega_a\omega_b}{(\omega_b-\omega_a)s}+1}
\end{equation}
On remplace alors le s par son équivalent en fréquence, soit:
\begin{equation}
s=\frac{2}{T_e}(\frac{z-1}{z+1})
\end{equation}
On obtient alors:
\begin{equation}
H(z)=\frac{1}{\frac{\frac{2}{T_e}(\frac{z-1}{z+1})^2+\omega_a\omega_b}{(\omega_b-\omega_a)\frac{2}{T_e}(\frac{z-1}{z+1})}+1}
\end{equation}
En simplifiant le tout pour ne garder qu'un numérateur et un dénominateur, on obtient:
\begin{equation}
H(z)=\frac{(\omega_b-\omega_a)(z-1)(z+1)}{\frac{4}{T_e^2}(z^2-2z+1)+\frac{2}{T_e}(\omega_b-\omega_a)(z-1)(z+1)+\omega_a\omega_b(z+1)^2}
\end{equation}
En sachant que $T_e = \frac{1}{F_e}$ et que $Fe$ vaut 8000, $Te = 1/8000$.
Afin de déterminer la forme finale du filtre, on utilise le premier filtre de Butterworth pour la fréquence de 500Hz avec une fenêtre de 400Hz à 600Hz. En les transformant en $\theta$ grâce à l'expression:
\begin{equation}
\theta = 2*arctan(\frac{\omega T_e}{2})
\end{equation}
Et ensuite la formule:
\begin{equation}
\omega = \frac{2}{T_e}*tan(\frac{\theta}{2})
\end{equation}
Les valeurs obtenues sont:
\begin{equation}
\omega_a = 2534,15
\end{equation}
\begin{equation}
\omega_b = 3841,26
\end{equation}

En remplaçant tous les termes calculés dans l'équation (6) et en additionnant tous les termes des mêmes ordres, on obtient l'équation finale de $H(z)$, soit:
\begin{equation}
H(z)=\frac{-0.0729573z^{-2} - 0.145915z^{-1} + 0.0729573}{0.854085z^{-2} - 1.86417z^{-1} + 1}
\end{equation}

\section{Coussin-trampoline}



\end{document}